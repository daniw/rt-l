\section{Teilaufgabe 9}
\begin{aufgabe}
    Eine Regelung (geschlossener Regelkreis) wird jetzt getestet (ohne 
    Vorsteuerung und Störgrössenaufschaltung). Zeichen Sie das Blockschaltbild 
    einer solchen Regelung (die 2 Übertragungsfunktionen $P_1(s)$ und 
    $P_2(s)$, die Laplace-Transformation der Referenzdrehzahl $\Omega_r(s)$ 
    und die Übertragungsfunktion des Reglers $C(s)$ sollten in diesem 
    Blockschaltbild ersichtlich sein).
\end{aufgabe}
\begin{figure}[h!]
    \centering
    \begin{tikzpicture}[node distance=2.0cm]
        \node (soll)    []                                  {$\Omega_r(s)$};
        \node (diff)    [sumpoint, right of=soll]           {};
        \node (control) [controlblock, right of=diff]       {$C(s)$};
        \node (path1)   [controlblock, right of=control]    {$P_1(s)$};
        \node (sum)     [sumpoint, right of=path1]          {};
        \node (feedback)[crosspoint, right of=sum]          {};
        \node (ist)     [right of=feedback]                 {$\Omega(s)$};
        \node (path2)   [controlblock, above of=sum]        {$P_2(s)$};
        \node (moment)  [above of=path2]                    {$\Gamma(s)$};
        \draw[thick, ->] (soll)     -- (diff) node[above left] {$+$};;
        \draw[thick, ->] (diff)     -- (control);
        \draw[thick, ->] (control)  -- (path1);
        \draw[thick, ->] (path1)    -- (sum) node[below left] {$+$};
        \draw[thick, ->] (sum)      -- (ist);
        \draw[thick, ->] (moment)   -- (path2);
        \draw[thick, ->] (path2)    -- (sum) node[above right] {$+$};
        \draw[thick, ->] (feedback) |- ++(0, -1.5cm) -| (diff) node[below right] {$-$};
    \end{tikzpicture}
    \caption{Blockschaltbild Regelung}
\end{figure}
