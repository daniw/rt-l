\section{Teilaufgabe 27}
\begin{aufgabe}
    Eine Vorsteuerung wird hinzugefügt. Zeichnen Sie jetzt das Blockschaltbild 
    des geregelten Prozesses. Die Übertragungsfunktion der Vorsteuerung wird 
    $F(s)$ genannt ($F$ steht für Feedforward auf English $\to$ Vorsteuerung 
    auf Deutsch)
\end{aufgabe}
\begin{figure}[h!]
    \centering
    \begin{tikzpicture}[node distance=2.0cm]
        \node (soll)    []                                  {$\Omega_r(s)$};
        \node (in)      [crosspoint, right of=soll]         {};
        \node (diff)    [sumpoint, right of=in]             {};
        \node (control) [controlblock, right of=diff]       {$C(s)$};
        \node (sum1)    [sumpoint, right of=control]         {};
        \node (path1)   [controlblock, right of=sum1]       {$P_1(s)$};
        \node (sum2)    [sumpoint, right of=path1]          {};
        \node (feedback)[crosspoint, right of=sum2]         {};
        \node (ist)     [right of=feedback]                 {$\Omega(s)$};
        \node (path2)   [controlblock, above of=sum2]       {$P_2(s)$};
        \node (moment)  [above of=path2]                    {$\Gamma(s)$};
        \node (forward) [controlblock, above of=control]    {$F(s)$};
        \draw[thick, ->] (soll)     -- (diff) node[above left] {$+$};;
        \draw[thick, ->] (diff)     -- (control);
        \draw[thick, ->] (control)  -- (sum1) node[below left] {$+$};
        \draw[thick, ->] (sum1)     -- (path1);
        \draw[thick, ->] (path1)    -- (sum2) node[below left] {$+$};
        \draw[thick, ->] (sum2)     -- (ist);
        \draw[thick, ->] (moment)   -- (path2);
        \draw[thick, ->] (path2)    -- (sum2) node[above right] {$+$};
        \draw[thick, ->] (feedback) |- ++(0, -1.5cm) -| (diff) node[below right] {$-$};
        \draw[thick, ->] (in)       |- (forward);
        \draw[thick, ->] (forward)  -| (sum1) node[above right] {$+$};
    \end{tikzpicture}
    \caption{Blockschaltbild Regelung}
\end{figure}
