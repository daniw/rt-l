\section{Teilaufgabe 8}
\begin{aufgabe}
    Programmieren Sie diese Steuerung mit Simulink. Wenn der Lastmoment gleich 
    null ist, ist die ideale Steuerung $S_t(s)$ das Inverse von $P_1(s)$ da 
    $S_t(s) \cdot {P_1}^{-1}(s) = 1$ (Referenzdrehzahl gleich Ist-Drehzahl). 
    Warum ist diese Steuerung nicht realisierbar? Anstatt zu ${P_1}^{-1}(s)$ 
    nehmen Sie einfach $\frac{1}{K_g}$ für $S_t(s)$ und testen Sie diese 
    Steuerung im Simulation.  Was passiert wenn der Wert von $K_g$ für die 
    Steuerung ($\frac{1}{K_g}$) nicht mehr der gleiche wie der Wert von $K_g$ 
    in der Übertragungsfunktion $P_1(s)$ ist? Was passiert wenn der Lastmoment 
    nicht null ist? Simulieren Sie diese 2 Fälle und zeigen Sie somit 2 von 
    der grossen Limitierung einer einfachen Steuerung im Vergleich zu einer 
    Regelung.
\end{aufgabe}
