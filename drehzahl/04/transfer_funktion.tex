\section{Übertragungsfunktionen}
Gegeben sind die folgenden Gleichungen für $u(t)$ und $\omega(t)$
\[
	u(t) = R \cdot i(t) + L \frac{di(t)}{dt} + K_\omega \cdot \omega(t)
\]
\[
	J \frac{d\omega(t)}{dt}
	= K_\omega \cdot i(t) - M_l - \alpha \cdot \omega(t)
\]
Aus der zweiten Gleichung lässt sich die Funktion für $i(t)$ explizit
aufstellen.
\[
	i(t)
	= \frac{J \cdot \frac{d\omega(t)}{dt} + M_l
		+ \alpha \cdot \omega(t)}{K_\omega}
\]
Dieses $i(t)$ lässt sich nun in die erste Gleichung einsetzen, somit ergibt
sich
\[
	u(t)
	= R \cdot \frac{J}{K_\omega} \cdot \frac{d\omega(t)}{dt} 
		+ R \cdot \frac{M_l}{K_\omega}
		+ R \cdot \frac{\alpha}{K_\omega} \cdot \omega(t)
		+ L \cdot \frac{J}{K_\omega} \cdot \frac{d^2\omega(t)}{dt^2}
		+ L \cdot \frac{\alpha}{K_\omega} \cdot \frac{d\omega(t)}{dt}
		+ K_\omega \omega(t)
\]
Diese Funktion für die Spannung $u(t)$ kann nun in den Bildbereich überführt
werden mit der Laplace-Transformation.
\[
	U(s)
	= R \cdot \frac{J}{K_\omega} \cdot \Omega(s)
		+ \underbrace{R \cdot \frac{M_l}{K_\omega}}_{0}
		+ R \cdot \frac{\alpha}{K_\omega} \cdot \Omega(s)
		+ L \cdot \frac{J}{K_\omega} \cdot \Omega(s) \cdot s^2
		+ L \cdot \frac{\alpha}{K_\omega} \cdot \Omega(s) \cdot s
		+ K_\omega \cdot \Omega(s)
\]
Der Term
\[ R \cdot \frac{M_l}{K_\omega} \]
beschreibt eine Störgrösse und hat nichts mit dem Eingangssignal zu tun. Somit
darf dieser Term gestrichen werden für die Übertragungsfunktion. Die
resultierende Funktion kann nun in eine günstige Form ungestellt werden, damit
der Quotient aufgestellt werden kann von Ausgangs- und Eingangssignal.
\[
	U(s)
	= \Omega(s) \cdot \left(
		\frac{L \cdot J}{K_\omega} \cdot s^2
		+ \frac{R \cdot J + L \cdot \alpha}{K_\omega} \cdot s 
		+ \frac{R \cdot \alpha}{K_\omega}
		+ K_\omega
	\right)
\]
Die Übertragungsfunktion lautet somit
\[
	G_1(s)
	= \frac{1}{ \frac{L \cdot J}{K_\omega} \cdot s^2
		+ \frac{R \cdot J + L \cdot \alpha}{K_\omega} \cdot s
		+ \frac{R \cdot \alpha}{K_\omega}
		+ K_\omega
	}
\]
