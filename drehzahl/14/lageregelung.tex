\section{Lageregelung}
Die Strecke besteht aus der Strecke für die Geschwindigkeit und einem 
zusätzlichen Integrator. Das heisst, die Strecke wird mit $\frac{1}{s}$ 
multipliziert. 
\[
    G_1(s) = \frac{1}{ \frac{L \cdot J}{K_\omega} \cdot s^3
        + \frac{R \cdot J + L \cdot \alpha}{K_\omega} \cdot s^2
        + \left(\frac{R \cdot \alpha}{K_\omega} + K_\omega\right) \cdot s
    }
\]
Somit ist beim Regler $C(s)$ kein Integrator mehr notwendig, um die Anforderung 
"'keine stationäre Ungenauigkeit"' zu erfüllen. \\
Für die Stabilität des Regelkreises wäre ein PI Regler sogar ungünstig, er 
eine Phasendrehung von $-90~[\si{\degree}]$ unterhalb von $\omega_e$ erzeugt. 

\noindent Um die Lageregelung in Betrieb zu nehmen, blieb leider nach 
Abschluss der restlichen Messungen keine Zeit mehr. \\
