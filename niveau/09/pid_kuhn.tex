\section{Dimensionierung PID Regler nach Kuhn}
Die PID-Reglereinstellung nach Kuhn gibt die folgende Auslegung vor
\[ K_p = \frac{2}{K_s} \qquad ,K_s = K_{g_n} \]
\[ T_i = 0.8 \cdot T_\Sigma \qquad ,T_\Sigma \xrightarrow{PT_1} \tau_n\]
\[ T_d = 0.194 \cdot T_\Sigma \qquad ,T_\Sigma \xrightarrow{PT_1} \tau_n \]
Die allgemeine Übertragungsfunktion zum PID-Regler lautet
\[
	\frac{U(s)}{E(s)}
	= K_p \left(
		1 + \frac{1}{T_i s}
		+ \frac{T_d s}{\left(\frac{T_d}{N}\right)s + 1}
	\right)
	\qquad , N = 10
\]
Mit den Einstellregeln nach Kuhn ergeben sich für die zwei
Übertragungsfunktionen $G_5(s)$ und $G_7(s)$ die folgenden Parameter.

\subsection{PID-Reglereinstellung für $G_5(s)$}
\[
	\begin{array}{l c l}
		K_p & = & 0.4116 \si[per-mode = fraction]{\volt\per\cm} \\
		T_i & = & 56 \si{\second} \\
		T_d & = & 13.58 \si{\second} \\
		N & = & 10
	\end{array}
\]

\subsection{PID-Reglereinstellung für $G_7(s)$}
\[
	\begin{array}{l c l}
		K_p & = & 0.3448 \si[per-mode = fraction]{\volt\per\cm} \\
		T_i & = & 78.4 \si{\second} \\
		T_d & = & 19.012 \si{\second} \\
		N & = & 10
	\end{array}
\]
